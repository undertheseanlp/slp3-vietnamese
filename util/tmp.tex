\section{Giới thiệu}

Trong quá trình xây dựng underthesea, tôi thường sử dụng rất nhiều thuật toán về xử lý ngôn ngữ tự nhiên.
Từ đơn giản như logistic regression, naive bayes đến các thuật toán hiện đại như xgboost, word embeddings, mạng neuron.
Tôi viết tài liệu này với mong muốn ngày càng hoàn thiện khả năng viết báo cáo kỹ thuật của mình, thuận tiện cho việc tìm hiểu và cập nhật thêm sau này.
Tôi cũng hi vọng nó hữu ích với các bạn sinh viên, các nhà nghiên cứu về lĩnh vực xử lý ngôn ngữ tự nhiên.

Cấu trúc của báo cáo được chia thành từng chương, ngoài chương đầu giới thiệu, mỗi chương sau sẽ trình bày các đặc điểm cần chú ý chính của  một thuật toán, hoặc một vấn đề liên quan đối với ngôn ngữ tự nhiên.
Đối với các vấn đề kỹ thuật sâu hơn, tôi viết kỹ hơn ở phần phụ lục, với mục đích đưa ra nhiều nền tảng toán hơn cho các bạn. Cũng như tránh được việc phải đọc quá nhiều cho bạn đọc.

\section{Bài toán phân loại}

Với $Y$ là tập các lớp, $X$ là tập các thông tin ngữ cảnh, nhiệm vụ của bài toán phân lớp là xây dựng một mô hình thống kê mà dự đoán chính xác lớp của văn bản bất kì.

Có hai cách tiếp cận để giải quyết bài toán này. Phương pháp thứ nhất là ước lượng xác suất đồng thời $p(y,x)$, phương pháp này được sử dụng trong các mô hình sinh (generative models)\footnote{\label{manning_maxent}\href{https://web.stanford.edu/class/cs124/lec/Maximum_Entropy_Classifiers.pdf}{Christopher Manning, Maxent Models and Discriminative Estimation}}. Các giải thuật dùng mô hình này như logistic regression, conditional loglinear, các mô hình maximum entropy và conditional random fields.

\noindent Phương pháp thứ hai là ước lượng xác suất có điều kiện $p(y|x)$, phương pháp này được sử dụng trong các mô hình phân biệt (discriminative models).
Ưu điểm của mô hình phân biệt là chúng đưa ra độ chính xác tốt hơn, dễ dàng tích hợp với các đặc trưng ngôn ngữ, và giúp xây dựng các mô hình xử lý ngôn ngữ tự nhiên độc lập với ngôn ngữ \footnoteref{manning_maxent}.



\section{Naive Bayes}

Là một phương pháp phân loại có giám sát. Dù rất dễ hiểu và dễ cài đặt, nhưng kết quả thu được lại rất tốt, vì thế đây là 1 phương pháp rất quan trọng trong NLP. Ứng dụng đầu tiên trong NLP của máy phân loại này là phân loại văn bản. Gần đây, máy phân loại này còn được ứng dụng thành công vào phần mềm lọc spam tự động. \footnote{\href{http://viet.jnlp.org/kien-thuc-co-ban-ve-xu-ly-ngon-ngu-tu-nhien/machine-learning-trong-nlp/phan-loai-van-ban-bang-dinh-ly-bayes}{http://viet.jnlp.org/kien-thuc-co-ban-ve-xu-ly-ngon-ngu-tu-nhien/machine-learning-trong-nlp/phan-loai-van-ban-bang-dinh-ly-bayes}}

\textbf{Biểu diễn bài toán}

Vector đặc trưng $x$ biểu diễn số lần xuất hiện các từ trong văn bản, $y$ là các nhãn mà văn bản thuộc về (ví dụ như thể thao, kinh tế, giải trí, ...)

Cho một tập dữ liệu đã được gán nhãn $D = \left{ (x^{(i)}, y^{(i}) \right}$ với $i=1 \sim N$

Ở đây $x^(i)$ là vector đặc trưng thứ $i$ trong tập huấn luyện, $y^(i)$ thuộc ${1, 2, ... , C}$ là các nhãn tương ứng với vector đó.
$x^{(i)} = \left( x^{(i)}_{1}, x^{(i)}_{2}, ..., x^{(i)}_{D}\right)$ là số lần xuất hiện của từ thứ d trong từ điển (từ giờ sẽ gắn số thứ tự với từ, nên sẽ gọi là từ $d$)

Áp dụng công thức Bayes, ta tính giá trị của $p(y|x)$, nếu giá trị này lớn hơn 1 giá trị $t$ cho trước, ta kết luận nhãn của vecotr $x$ là $y$.

Yêu cầu đặt ra là ngăn chặn spam bằng cách phân loại một email gửi đến là spam hay non-spam. Cần đạt được hiệu quả phân loại thật khả quan.
Tuy nhiên cần tuyệt đối tránh lỗi sai chằng non-spam là spam vì có thể gây hiêu quả nghiệm trọng hơn là khả năng lọc spam thấp.

\section{MaxEnt}

Đối với các bài toán phân lớp dữ liệu, entropy cực đại là một kỹ thuật để ước lượng xác suất các phân phối từ dữ liệu.
\footnote{\href{http://viet.jnlp.org/kien-thuc-co-ban-ve-xu-ly-ngon-ngu-tu-nhien/machine-learning-trong-nlp/thuat-toan-entropy-cuc-dai}{http://viet.jnlp.org/kien-thuc-co-ban-ve-xu-ly-ngon-ngu-tu-nhien/machine-learning-trong-nlp/thuat-toan-entropy-cuc-dai}}
Tư tưởng chủ đạo của nguyên lý entropy cực đại là "mô hình phân phối đối với mỗi tập dữ liệu và tác ràng buộc đi cùng phải đạt được độ cân bằng đều nhất có thể".
Những ràng bày được thể hiện bởi các giá trị ước lượng được của các đặc trưng. Từ các ràng buộc sinh ra bởi tập dữ liệu này, mô hình sẽ tiến hành tính toán để có được một phân phối entropy cực đại.

\input{tex/conditional_random_fields.tex}

\section{Word Embeddings}

\textbf{Word Embeddings là gì?}

Word Embeddings là một phương pháp biểu diễn từ (hay tổng quát hơn, các đối tượng trong từ điển) thành vector.

\textbf{Có những phương pháp word embeddings nào?}

Cùng với sự phát triển mạnh mẽ của deep learning, nhiều phương pháp word embeddings \footnote{\href{https://pydata.org/london2018/schedule/presentation/29/}{Beyond word2vec: GloVe, fastText, StarSpace. Konstantinos Perifanos. pydata 2018}} ra đời như word2vec \cite{DBLP:conf/nips/MikolovSCCD13}, GloVe \cite{DBLP:conf/emnlp/PenningtonSM14}, RAND-WALK \cite{DBLP:journals/corr/AroraLLMR15}, fastText \cite{DBLP:conf/eacl/GraveMJB17}, StarSpace \cite{DBLP:conf/aaai/WuFCABW18} và các phương pháp khác.

Trong đó, fastText gây ấn tượng bằng việc cung cấp một thư viện huấn luyện cực kì nhanh. Các corpus khoảng 100000 văn bản chỉ cần huấn luyện trong vòng vài giây. word2vec được cho là phương pháp tiên phong, trong các báo cáo về sử dụng deep learning cho văn bản, thường sử dụng các embedded vector cho các từ được huấn luyện bởi word2vec trong các input layer.

\section{word2vec}

word2vec là một phương pháp biểu diễn từ thành vector, được tác giả Tomas Mikolov và cộng sự giới thiệu trong báo cáo tại hội nghị NIPS năm 2013 \cite{DBLP:conf/nips/MikolovSCCD13}.

Hai thuật toán để học mô hình word2vec là continous bag-of-words (CBOW) và continous skip-gram. Tư tưởng chính của skip-gram là từ một từ dự đoán ngữ cảnh của từ đó.

word2vec đưa ra các phương pháp để cải thiện hiệu năng của hệ thống như negative sampling, hierarchical softmax, async SGD.

\section{GloVe}

Glove \cite{DBLP:conf/emnlp/PenningtonSM14} sử dụng kỹ thuật phân tích ma trận. Cho một corpus đầu vào, thuật toán xây dựng ma trận co-occurrence của các từ trong một ngữ cảnh.

\section{fastText}

fastText \cite{DBLP:conf/eacl/GraveMJB17} là một thuật toán mở rộng từ word2vec, được tác giả Armand Joulin  và cộng sự giới thiệu trong báo cáo vào năm 2016.

\noindent Sự khác biệt giữa fastText và word2vec? \footnote{\href{https://www.quora.com/What-is-the-main-difference-between-word2vec-and-fastText}{https://www.quora.com/What-is-the-main-difference-between-word2vec-and-fastText}}

\textit{word2vec coi các từ trong corpus như một phần tử độc lập và sinh ra một vector cho mỗi từ}

\noindent Trong khi đó,

\textit{fastText coi mỗi từ là một tập hợp các các ký tự ngrams. Nên một vector của một từ là tổng của các ký tự n grams của nó}

Mô hình của fastText là từ -> thông tin subword -> n-grams -> embedding cho các n-grams

Ví dụ, vector cho từ "apple" là trung bình của các vector của các n-gram "<ap", "app", "apple", "apple", "apple>", "ppl", "pple", "pple>", "ple", "ple>", "le>" (giả sử cấu hình cho ngrams min là 3 và nagrams max là 6).

\section{StarSpace}

StarSpace được đề xuất bởi nhóm tác giả Ledell Yu Wu và cộng sự. Star là cách đọc của *, với ý tưởng là có thể sinh ra embedding vector cho mọi đối tượng, dựa vào ngữ cảnh của nó.

Mục tiêu của StarSpace là tối ưu hóa hàm

$$J = \sum L^{batch} \left( sim(a, b), sim(a, b_1^{-}),..., sim(a, b_n^{-}) \right)$$

trong đó,

\begin{itemize}
  \item Sinh ra tập các cặp dương (a, b), từ tập $E^{+}$
  \item Sinh ra các đối tượng âm $E^{-}$ (với phương pháp negative sampling giống word2vec)
  \item sim: độ đo tương đồng (sử dụng độ đo cosine hoặc dot product)
  \item $L_{batch}$ so sánh giữa cặp dương và các cặp âm (hinge, softmax)
\end{itemize}

\bibliography{technique_report}
\bibliographystyle{acl_natbib}

\clearpage

\begin{appendices}
\section{Naive Bayes}
Sử dụng Naive Bayes trong bài toán phân loại tin nhắn rác. Tập dữ liệu SAMPLE SAMPLE 20180529 được trình bày trong bảng \ref{table:spam_sample}.

\begin{table*}
  \centering
  \begin{tabular}{lc}
    \toprule
    Tin nhắn & Phân loại \\
    \midrule
    nhà trung cư giá chỉ 2.3 tỷ & RÁC \\
    giảm giá đi du lịch , đăng ký ngay hôm nay & RÁC \\
    iphone giá rẻ mua trả góp & RÁC \\
    hôm nay đi làm vui lắm anh ạ & BÌNH THƯỜNG \\
		anh hôm nay về nhà sớm, đi chơi nhé & BÌNH THƯỜNG \\
		nhớ mua hộ em quyển sách nhé & BÌNH THƯỜNG
    \bottomrule
  \end{tabular}


  \caption{\small Dữ liệu tin nhắn rác SPAM SAMPLE 20180529}
  \label{table:spam_sample}
\end{table*}

\section{Maximum Entropy}

Mô hình entropy cực đại cung cấp một phương pháp đơn giản để ước lượng xác suất có điều kiện $p(y|x)$ thông qua việc thống kê các thuộc tính quan trọng quan sát được từ tập dữ liệu huấn luyện \footnote{\href{http://uet.vnu.edu.vn/~thuyhq/Student_Thesis/K47_Nguyen_Thi_Thuy_Linh_Thesis.pdf}{Nguyễn Thị Thùy Linh, Phân lớp tài liệu web độc lập ngôn ngữ}}.

\noindent \textbf{Tập dữ liệu huấn luyện} Để làm bài toán phân lớp, trước tiên phải xây dựng tập dữ liệu $T = \{ (x_1, y_1), ... , (x_N, y_N) \}$, trong đó \{x_1, ... , x_N\}$ là tập các thông tin ngữ cảnh đã được gán nhãn tương ứng là tập các lớp $\{y_1,..., y_N\}$

Với mỗi cặp $(x_i, y_i)$, phân phối xác suất thực nghiệm của nó được tính bởi

$\widetilde{p} (x_i, y_i) = \frac{1}{N} \times $ số lần xuất hiện của $(x_i, y_i)$ trong tập dữ liệu mẫu

Thông thường thì mỗi cặp $(x_i, y_i)$ không thể không xuất hiện trong tập mẫu, mà nó sẽ xuất hiện ít nhất một lần.

%TODO: TO BE UPDATED Feature Indicator, Solving Algorithms






\end{appendices}